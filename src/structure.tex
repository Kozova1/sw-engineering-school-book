\chapter{מבנה האפליקציה}

הפתרון שבניתי הוא אפליקציה אשר מתפקדת בדומה ל-\witheng{Google Classroom},
אך כוללת כלים המיועדים ספציפית ללימוד שפת תוכנה.
האפליקציה מאפשרת שימוש בשפת Lua על מנת להגדיר תרגילים ובדיקה אוטומטית שלהם.

\section{מחלקות}

\paragraph{Article}

מחלקה המחזיקה את כל המידע של שיעור אחד.
מתפקדת גם כאחת מהישויות בבסיס הנתונים.

\paragraph{Exercise}

מחלקה המחזיקה את כל המידע של מטלה / תרגיל אחד.
מתפקדת גם כאחת מהישויות בבסיס הנתונים.

\paragraph{Course}

מחלקה המחזיקה קורס שלם - נמצאת בשימוש רק על מנת להקל על המרה ל-JSON ושליחה לבסיס הנתונים.

\paragraph{ListFragment}

מחלקה המציגה למשתמש את רשימת חומרי הלימוד (שיעורים) או המטלות בקורס הנוכחי,
המחלקה משתמשת ב-RecyclerView על מנת להציג את הרשימה.
בהתאם למידע שהועבר ב-Intent.
על מנת להזין את המידע ל-RecyclerView המחלקה משתמשת במחלקות ExerciseListAdapter ו-ArticleListAdapter.

\paragraph{CreatingArticlesInfoFragment}

מחלקה המציגה למשתמש מידע בסיסי על יצירת חומרי לימוד (שיעורים) בקורס.
משתמשת בספריית Markwon על מנת להמיר קוד Markdown לטקסט קריא.

\paragraph{CreatingExercisesInfoFragment}

מחלקה המציגה למשתמש מידע על יצירת מטלות בקורס,
כולל הסברים על כתיבת בדיקות אוטומטיות לתרגילים.
משתמשת בספריית Markwon על מנת להמיר קוד Markdown לטקסט קריא.

\paragraph{Listing}

מחלקה המייצגת כרטיס אחד, אשר יכול לייצג מטלה או שיעור (חומר לימוד).
המחלקה מכילה את הפונקצייה bindTo, אשר מקבלת ערך מסוג T extends Listable ומעתיקה את המידע בו לעצמה, כך שתתאים לשיעור או למטלה שאותה היא מייצגת.


\paragraph{Listable}

ממשק המייצג או Article או Exercise, נמצא בשימוש רק כדי להגביל את מחלקת Listing לשתי האפשרויות האלו.

\paragraph{ReplFragment}

מחלקה המציגה למשתמש REPL המאפשר לו להריץ קוד בשפת Lua.
המחלקה משתמשת בספריית LuaJ על מנת להריץ קוד Lua.
הספרייה משמשת להרצת הפתרונות ולבדיקתם.
על מנת לדאוג שהפלט יופיע בצורה נכונה, משתמשים ב-PipedInputStream וב-PipedOutputStream בתור stdout (פלט ראשי) של המכונה הווירטואלית שמריצה את קוד ה-Lua.
המכונה הווירטואלית מכילה מספר ספריות מובנות, כמו ספריית IO, ספריית coroutines ועוד.
המכונה הווירטואלית מכילה גם את ספריית java, המאפשרת לקוד Lua לתקשר עם קוד Java.
הספרייה הזאת אינה משולבת באפליקציה מסיבות אבטחה - נתינת גישה לקוד Java עלול לגרום לנזק לאפליקציה.

\paragraph{ArticleFragment}

מחלקה המציגה למשתמש שיעור אחד, אשר מפורמט באמצעות שפת Markdown.
המחלקה משתמשת בספריית Markwon על מנת להמיר קוד Markdown לטקסט, כך שיהיה ניתן להציג אותו.
הספרייה דואגת לפרמוט הטקסט, גודל, גופן ועוד.
המחלקה משתמשת גם בספריית Prism4j, על מנת לדאוג שקוד ה-Markdown יופיע בצבעים הנכונים.

בנוסף לכך, המחלקה משתמשת בספריית LuaJ על מנת להריץ קוד Lua.
הספרייה משמשת להרצת הפתרונות ולבדיקתם.

\paragraph{ExerciseFragment}

מחלקה המציגה למשתמש מטלה אחת, עם הוראות מפורמטות באמצעות Markdown ותיבת טקסט להזנת קוד.
המחלקה כוללת FloatingActionButton אשר משמש להרצת הקוד ולבדיקתו.
המחלקה משתמשת בספריית Markwon על מנת להמיר קוד Markdown לטקסט, כך שיהיה ניתן להציג אותו.
הספרייה דואגת לפרמוט הטקסט, גודל, גופן ועוד.
המחלקה משתמשת גם בספריית Prism4j, על מנת לדאוג שקוד ה-Markdown יופיע בצבעים הנכונים.


\paragraph{AddContentFragment}

מחלקה המקבלת את בחירת המשתמש אם להוסיף שיעור או מטלה, ומדליק את ה-Fragment הרלוונטי.

\paragraph{AddArticleFragment}

מחלקה המציגה למשתמש את מסך הוספת שיעור (או חומר לימוד) לקורס.

\paragraph{AddExerciseFragment}

מחלקה המציגה למשתמש את מסך הוספת מטלה לקורס.

\paragraph{ArticleListAdapter}

מחלקה המזינה נתונים על שיעורים וחומרי לימוד שונים מבסיס הנתונים המקומי (Room) ל-ListFragment.

\paragraph{ExerciseListAdapter}

מחלקה המזינה נתונים על מטלות שונות מבסיס הנתונים המקומי (Room) ל-ListFragment.

\section{מסכים ופעולות - Activites}

\subsection*{SplashScreenActivity}

מציג מסך חד פעמי המסביר למשתמש על האפליקציה ועל שפת Lua בפעם הראשונה שהאפליקציה נפתחת.

\paragraph{moveToNextActivity}

הפעולה עוברת ל-FirstTimeActivity ונסגרת, בלי להשאיר את עצמה על ה-\witheng{Back Stack}.

\paragraph{wasSplashShownBefore}

הפונקצייה בודקת ב-SharedPreferences האם הפעולה הורצה בעבר.

\begin{itemize}
  \item פרמטר \witheng{Activity act} - הפעולה שבודקים ב-SharedPreferences שלה.
\end{itemize}

\paragraph{setSplashShown}

הפונקצייה רושמת ב-SharedPreferences שהפעולה הורצה בעבר, על מנת שלא תודלק שוב.

\begin{itemize}
  \item פרמטר \witheng{Activity act} - הפעולה שרושמים ב-SharedPreferences שלה.
\end{itemize}

\subsection*{FirstTimeActivity}

מציג מסך שבו מחליטים אם ליצור קורס או להכנס לקורס קיים, בעזרת קוד בן 8 ספרות הניתן בפרסום הקורס.
הקוד מייצג ערך השמור בבסיס הנתונים החיצוני (Firebase RTDB), אשר מורד ומומר מ-JSON לערך מסוג Course.
אם המשתמש כבר נמצא בקורס, הפעולה עוברת ל-MainActivity.

\paragraph{moveToNextActivity}

הפעולה עוברת ל-MainActivity ונסגרת, בלי להשאיר את עצמה על ה-\witheng{BackStack}.

\paragraph{isInCourse}

בודק האם המשתמש נמצא כרגע בקורס לפי ערך ה-SharedPreferences שלו.

\begin{itemize}
  \item פרמטר \witheng{Context ctx} - ההקשר שממנו מקבלים את ה-SharedPreferences.
\end{itemize}

\paragraph{beginCourse}

כותב ל-SharedPreferences שהמשתמש התחיל קורס.

\begin{itemize}
  \item פרמטר \witheng{Context ctx} - ההקשר שממנו מקבלים את ה-SharedPreferences.
\end{itemize}

\paragraph{endCourse}

כותב ל-SharedPreferences שהמשתמש יצא מהקורס.

\begin{itemize}
  \item פרמטר \witheng{Context ctx} - ההקשר שממנו מקבלים את ה-SharedPreferences.
\end{itemize}

\subsection*{ViewerActivity}

מציג למשתמש שיעור (Article) או תרגיל (Exercise), באמצעות ExerciseFragment או ArticleFragment.

\paragraph{onOptionItemSelected}

דורס פעולה של Activity.
בודק האם המשתמש לחץ על כתפור ``חזור'', ואם כן חוזר ל-MainActivity.

\begin{itemize}
  \item פרמטר \witheng{@NonNull MenuItem item} - ה-MenuItem שנבחר על ידי המשתמש.
  \item ערך מוחזר boolean - האם ה-MenuItem שנבחר זוהה.
\end{itemize}

\subsection*{MainActivity}

הפעולה המרכזית, המכילה \witheng{Navigation Drawer} אשר בוחר איזה Fragment מוצג.

\paragraph{airplaneNagger}

משתנה מסוג BroadcastReceiver הבודק האם הטלפון מחליף למצב טיסה,
ואם כן שולח Notification שמציע למשתמש להתקדם בחומר הלימוד באפליקציה.
ניתן לכבות ולהדליק את ה-BroadcastReceiver באמצעות אחת האפשרויות ב-Options Menu של MainActivity.

\paragraph{finishCourse}

כותב ל-SharedPreferences שהקורס נגמר באמצעות FirstTimeActivity.endCourse,
ואז חוזר ל-FirstTimeActivity באמצעות Intent.

\paragraph{onBackPressed}

דורס את Activity.onBackPressed על מנת לדאוג שה-\witheng{Navigation Drawer} נסגר כאשר חוזרים אחורה.

\paragraph{onCreateOptionsMenu}

דורס את Activity.onCreateOptionsMenu על מנת לדאוג להדליק ולכבות את ה-BroadcastReceiver בהתאם להגדרות השמורות.
בודק ב-SharedPreferences על מנת להחליט האם ה-Checkbox של ``חסום הודעות במצב טיסה'' צריכה להיות מסומנת.

\paragraph{onOptionsItemSelected}

דורס את Activity.onOptionsItemSelected על מנת לנהל את ה-\witheng{Options Menu} של הפעולה:

\begin{itemize}
  \item אם נבחר Reset Local Database - להעלות Alert Dialog, ואם נבחר yes - למחוק את כל התוכן של הקורס הנוכחי.
  \item אם נבחר Publish Course - להעתיק את כל המידע מבסיס הנתונים לתוך Course, להמיר ל-JSON, ליצור קוד בעל 8 ספרות רנדומליות המייצג את הקורס, לעתיק אותו ל-Clipboard ולהעלות ל-Firebase.
    לאחר הבחירה, הקורס נסגר באמצעות finishCourse.
  \item אם נבחר Exit Course - לסגור את הקורס באמצעות finishCourse.
  \item אם Disable Airplane Notification מסומן, להדליק את AirplaneModeDuolingoNotificationSender. אם לא מסומן, לכבות את ה-BroadcastReceiver.
\end{itemize}

\paragraph{setMenuItemWhite}

מוודא צבע הטקסט בפרמטר הוא לבן, על מנת שישאר קריא.

\begin{itemize}
  \item פרמטר MenuItem item - ה-MenuItem שיהפוך ללבן.
\end{itemize}

\subimport{structure}{flow}
\newpage
\subimport{structure}{classes}
