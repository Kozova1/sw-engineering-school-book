\chapter{מדריך למשתמש}
\section{מטרת האפליקציה}

%ימינה
\projectname היא אפליקציה המתפקדת בתור פלטפורמת לימוד לשפת התכנות Lua.
האפליקציה מאפשרת בדיקה אוטומאית של מטלות, הרצת קוד, שימוש בספריות ועוד.

\section{יכולות האפליקציה}

\begin{itemize}

  %ימינה
\item יצירת קורסים וכניסה אליהם באמצעות קוד בן 8 ספרות
\item בדיקה אוטומטית של תשובות של תלמידים למטלות באמצעות הרצת קוד Lua כבדיקה.
\item תמיכה בשמירה חיצונית של קורסים
\item שליחת הודעות במקרה של מעבר למצב טיסה המזכירות לתלמידים ללמוד באפליקציה.

\end{itemize}

\section{מדריך לתלמיד}

\newcommand{\screen}[3]{
  \begin{minipage}{0.45\linewidth}
    \begin{figure}[H]
        \centering
        \caption{#2}
        \includegraphics[width=\linewidth]{screenshots/#1}
    \end{figure}
    #3
  \end{minipage}
}

\screen{splash}{מסך הסבר בסיסי}{לאחר שקראתם את ההסבר, לחצו על הכפתור}
\hfill
\screen{firsttime}{מסך כניסה לקורס}{הזינו את מספר הקורס שקיבלתם מהמורה}

\screen{drawer}{תפריט המעבר}{בחרו אחת מהאפשרויות בתפריט והתחילו ללמוד!}
\hfill
\screen{repl}{לולאת הרצת קוד - REPL}{זהו ה-REPL. הוא נותן לכם להריץ קוד Lua ולראות את התוצאות. נסו אותו!}

\screen{exerciselist}{רשימת המטלות}{זוהי רשימת המטלות שלכם. למדו והשלימו את כולן!}
\hfill
\screen{articlelist}{רשימת השיעורים}{זוהי רשימת השיעורים שלכם. אם משהו לא ברור לכם במטלה, נסו לבדוק אם אחד מהשיעורים מכסה את החומר!}

\screen{article}{שיעור לדוגמה}{זהו שיעור דוגמה. כך נראים שיעורים מבפנים!}
\hfill
\screen{exercise}{מטלה לדוגמה}{זהו תרגיל דוגמה. על מנת לפתור אותו,עליכם לכתוב קוד בתיבת בטקסט אשר ממלא את הדרישות של המטלה.}

\screen{optionsmenu}{הגדרות שונות}{כאן ניתן לראות את ההגדרות. זהירות! חלק מהאפשרויות עלולות למחוק לכם את כל התשובות!}
\hfill
\screen{airplanemsg}{הודעת מצב טיסה}{אם אתם במצב טיסה, אולי כדאי שתנסו לפתור כמה מטלות?}

\section{מדריך למורה}

\screen{splash}{מסך הסבר בסיסי}{קראו את ההסבר ואז לחצו על הכפתור}
\hfill
\screen{firsttime}{מסך יצירת קורס}{לחצו על כפתור ``צור קורס''. תלמידיכם ילחצו על כפתור ``כנס לקורס''.}

\begin{center}
\screen{drawer}{תפריט}{ראשית כל, לחצו על שתי אפשרויות העזרה (השתיים התחתונות)}
\end{center}

\screen{creatingarticles}{מסך עזרה לגבי יצירת שיעורים}{המסך הזה יסביר לכם על יצירת שיעורים}
\hfill
\screen{creatingexercises}{מסך עזרה לגבי יצירת מטלות}{המסך הזה יסביר לכם על יצירת מטלות}

לאחר שקראתם את שני ההסברים, עברו למסך יצירת המטלות או השיעורים באמצעות בחירת אפשרות ``Add Articles or Exercises'' בתפריט.

\screen{createarticle}{מסך יצירת שיעור}{הזינו כאן את כותרת ואת תוכן השיעור. אם השיעור קשה במיוחד, סמנו במקום המתאים. תוכן השיעור יכול להשתמש בקוד Markdown למען פרמוט הטקסט.}
\hfill
\screen{createexercise}{מסך יצירת מטלה}{הזינו כאן את כותרת, הוראות, קוד בדיקה ואת הקוד ההתחלתי של המטלה. אם המטלה קשה במיוחד, סמנו זאת במקום המתאים. למען פרמוט ההוראות ניתן להשתמש בקוד Markdown. קוד הבדיקה הוא בשפת Lua, וצריך להכיל פונקצייה בשם test אשר לא לוקחת פרמטרים ומחזירה true אם הבדיקה הצליחה, או קורסת או מחזירה false במקרה אחר.}

\let\screen\undefined