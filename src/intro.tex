\newenvironment{PotentialSolution}[1]
{
  \newcommand{\solution}{\witheng{#1}}
  \paragraph{#1}
}{}

\newenvironment{PotentialSolutionHeb}[1]
{
  \newcommand{\solution}[0]{#1}
  \paragraph{#1}
}{}

\chapter{מבוא}

בתור תלמיד בתיכון, גיליתי מחסור באפשרויות לימוד לשפות תכנות. \\
קיים מגוון רב של אפליקציות הבנויות ללימוד בכיתה, אך אף אחת מהן לא תומכת בצורה פשוטה בהרצת קוד,
תרגילים שהתשובה להם היא כתיבת קוד ועוד.
התגלתה בעיה בדרך הבדיקה של שאלות עם קוד - המורים נאלצים לבדוק את הקוד ידנית, להזין אותו למחשב ולבדוק שאכן פועל באופן נכון.
הבעיה גרמה לי לחקור על פתרונות אחרים בתחום.

\section{פתרונות אחרים לבעיה}
\begin{PotentialSolution}{Google Classroom}

%ימינה
מאפשר למורים ליצור כיתות, שאליהן יכולים להצטרך תלמידים.
המורים יכולים להעלות תרגילים וחומרי למידה לתלמידים, ולקבוע תאריכי הגשה למטלות שונות.
\solution לא פותר את בעיית בדיקת התרגילים - המורה עדיין צריך/ה לבדוק ידנית כל אחת מהתשובות.
בנוסף לכך, \solution לא מכיל סביבת עבודה נוחה לכתיבת קוד.

\end{PotentialSolution}

\begin{PotentialSolution}{Code Academy}

% ימינה
מלמד תלמידים לתכנת במגוון שפות ללא נסיון קודם. מלמד באמצעות פתרון בעיות.
\solution לא מכיל כלים למורים, ולא ניתן לתת בו מטלות.
  
\end{PotentialSolution}

\subsection{חידושים בפרויקט}

הפרוייקט שלי הוא האפליקציה היחידה הכוללת בדיקת קוד אוטומטית,
עריכת חומרי לימוד בצורה פשוטה, כלים למורים וקונסולה להרצת קוד בשפה המלומדת.

\section{הפרויקט}

הפרויקט שלי הוא אפליקצית \projectname.
האפליקציה מלמדת תכנות באמצעות שפת Lua.
האפליקציה מאפשרת יצירת קורסים, אשר מכילים חומרי לימוד ומטלות לביצוע, אשר אינן מוגבלות בזמן.
לאחר שקורס נוצר, לא ניתן לשנות אותו.
כאשר יוצרים מטלה בקורס, יש להוסיף קוד Lua אשר בודק האם הפתרון שהוגש תקין.
האפליקציה כוללת REPL שבו ניתן לנסות קוד Lua, על מנת להקל על התלמידים בפתירת המטלות.
האפליקציה מאפשרת שימוש בשפת Markdown על מנת ליצור תרגילים וחומרי לימוד שנראים טוב.