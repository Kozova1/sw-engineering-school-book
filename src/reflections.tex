\chapter{רפלקצייה אישית}

העבודה על הפרוייקט הייתה מאוד מעניינת ומהנה.
למדתי ממנה את עקרונות פיתוח האפליקציות לאנדרואיד,
שיפרתי את ידע ה-Java שלי ולמדתי כיצד לעבוד עם ספריות חיצוניות בצורה הכי יעילה.

אחד הקשיים שבהם נתקלתי היה מציאת שפת תכנות שיכולה לרוץ בתוך Java.
שקלתי מספר אפשרויות כמו Jython (גרסה של פייתון ל-JVM), MRuby, אך כל אלו המירו Bytecode שלהן לBytecode של הJVM, מה שגרם לבעיות עם הJVM של אנדרואיד.
לבסוף החלטתי להשתמש בספריית LuaJ על מנת להריץ את שפת התכנות Lua.

אחד האתגרים האחרים היה להבין כיצד לדאוג שפונקציות שמשתמשות ב-IO בשפת Lua יעבדו.
הפוקנציות האלו דורשות Terminal שאליו הן יכולות לפלוט טקסט, אך אנדרואיד לא כולל Terminal.
לאחר בדיקת האפשרויות, בחרתי להשתמש ב-PipedStream על מנת ליצור מעין Terminal פרימיטיבי,
אשר מספיק כדי שפונקציות הדפסה וקליטה בסיסיות יעבדו.

אילו הייתי מתחיל את הפרוייקט עם הידע שהשגתי ממנו, הייתי מוסיף עורך קוד יותר חכם.
כרגע, עורך הקוד הוא פשוט EditText עם מספר שורות.
אילו היה לי עוד זמן, הייתי מוסיף צביעת קוד, הוספת רווחים (Indentation) אוטומטית ועוד.
הייתי שוקל גם להוסיף תמיכה לעוד שפות תכנות, למשל JavaScript (באמצעות ספריית Rhino של Mozilla).