\chapter{בסיסי נתונים}

\section{בסיס נתונים מקומי - Room Persistent Database}

\subsection{ישויות}

בסיס הנתונים המקומי מכיל שתי ישויות, כאשר כל אחת נשמרת בטבלה נפרדת עם השם שלה.
הישויות הן:

\paragraph{Article}

הישות מייצגת שיעור אחד, ולמעשה זהה למחלקת Article.

\paragraph{Exercise}

הישות מייצגת מטלה אחת, ולמעשה זהה למחלקת Exercise.

\subsection{טבלאות}

כל הערכים בבסיס הנתונים הם חובה (NOT NULL), על מנת למנוע שילובים בלתי הגיוניים,
למשל שיעור ללא תוכן או שיעור ללא כותרת.

\subsubsection*{Article}
\paragraph{uid}

ערך שלם, מפתח ראשי ובעל העלאה אוטומטית (AUTOINCREMENT) מייצג את מספר השיעור.

\paragraph{Title}

ערך מחרוזת, מכיל את הכותרת ואת שם השיעור.

\paragraph{ArticleContents}

ערך מחרוזת, מכיל את קוד ה-Markdown של תוכן השיעור.

\paragraph{HasBeenRead}

ערך שלם, מכיל ערך 0 או 1 (בוליאני, true או false בהתאם) בהתאם להאם השיעור הושלם (נקרא).

\noindent\rule{\textwidth}{0.4pt}

\subsubsection*{Exercise}
\paragraph{uid}

ערך שלם, מפתח ראשי ובעל העלאה אוטומטית (AUTOINCREMENT) מייצג את מספר התרגיל.

\paragraph{Title}

ערך , מכיל את הכותרת ואת שם המטלה.

\paragraph{Test}

ערך מחרוזת, מכיל את קוד ה-Lua שמוודא האם הפתרון תקין.

\paragraph{Instructions}

ערך מחרוזת, מכיל את קוד ה-Markdown אשר מוצג למשתמש כהוראות שימוש.

\paragraph{Template}

ערך מחרוזת, מכיל את קוד ה-Lua שמוצג למשתמש לפני שכתב קוד.

\paragraph{IsDone}

ערך שלם, מכיל ערך 1 או 0 (בוליאני, true או false בהתאם) בהתאם להאם התרגיל נפתר בהצלחה.

\paragraph{IsHard}

ערך שלם, מכיל ערך 1 או 0 (בוליאני, true או false בהתאם) בהתאם להאם התרגיל מסומן כקשה במיוחד.

\noindent\rule{\textwidth}{0.4pt}

\section{בסיס נתונים חיצוני - Firebase Realtime Database}
\subsection*{learnprogramming-lua-app-default-rtdb}

מאפשר לכל אחד להוסיף ולקרוא, אך לא למחוק או לשנות.
לא דורש כניסה על מנת להוסיף או לקרוא ערכים.
מכיל צמדי ערכים:

\paragraph{key} מספר בעל 8 ספרות, משמש כמספר הקורס.
\paragraph{value} גרסה מומרת ל-JSON של המחלקה Course.